%% Beginning of file 'sample631.tex'
%%
%% Modified 2021 March
%%
%% This is a sample manuscript marked up using the
%% AASTeX v6.31 LaTeX 2e macros.
%%
%% AASTeX is now based on Alexey Vikhlinin's emulateapj.cls 
%% (Copyright 2000-2015).  See the classfile for details.

%% AASTeX requires revtex4-1.cls and other external packages such as
%% latexsym, graphicx, amssymb, longtable, and epsf.  Note that as of 
%% Oct 2020, APS now uses revtex4.2e for its journals but remember that 
%% AASTeX v6+ still uses v4.1. All of these external packages should 
%% already be present in the modern TeX distributions but not always.
%% For example, revtex4.1 seems to be missing in the linux version of
%% TexLive 2020. One should be able to get all packages from www.ctan.org.
%% In particular, revtex v4.1 can be found at 
%% https://www.ctan.org/pkg/revtex4-1.

%% The first piece of markup in an AASTeX v6.x document is the \documentclass
%% command. LaTeX will ignore any data that comes before this command. The 
%% documentclass can take an optional argument to modify the output style.
%% The command below calls the preprint style which will produce a tightly 
%% typeset, one-column, single-spaced document.  It is the default and thus
%% does not need to be explicitly stated.
%%
%% using aastex version 6.3
\documentclass[twocolumn]{aastex631}

%% The default is a single spaced, 10 point font, single spaced article.
%% There are 5 other style options available via an optional argument. They
%% can be invoked like this:
%%
%% \documentclass[arguments]{aastex631}
%% 
%% where the layout options are:
%%
%%  twocolumn   : two text columns, 10 point font, single spaced article.
%%                This is the most compact and represent the final published
%%                derived PDF copy of the accepted manuscript from the publisher
%%  manuscript  : one text column, 12 point font, double spaced article.
%%  preprint    : one text column, 12 point font, single spaced article.  
%%  preprint2   : two text columns, 12 point font, single spaced article.
%%  modern      : a stylish, single text column, 12 point font, article with
%% 		  wider left and right margins. This uses the Daniel
%% 		  Foreman-Mackey and David Hogg design.
%%  RNAAS       : Supresses an abstract. Originally for RNAAS manuscripts 
%%                but now that abstracts are required this is obsolete for
%%                AAS Journals. Authors might need it for other reasons. DO NOT
%%                use \begin{abstract} and \end{abstract} with this style.
%%
%% Note that you can submit to the AAS Journals in any of these 6 styles.
%%
%% There are other optional arguments one can invoke to allow other stylistic
%% actions. The available options are:
%%
%%   astrosymb    : Loads Astrosymb font and define \astrocommands. 
%%   tighten      : Makes baselineskip slightly smaller, only works with 
%%                  the twocolumn substyle.
%%   times        : uses times font instead of the default
%%   linenumbers  : turn on lineno package.
%%   trackchanges : required to see the revision mark up and print its output
%%   longauthor   : Do not use the more compressed footnote style (default) for 
%%                  the author/collaboration/affiliations. Instead print all
%%                  affiliation information after each name. Creates a much 
%%                  longer author list but may be desirable for short 
%%                  author papers.
%% twocolappendix : make 2 column appendix.
%%   anonymous    : Do not show the authors, affiliations and acknowledgments 
%%                  for dual anonymous review.
%%
%% these can be used in any combination, e.g.
%%
%% \documentclass[twocolumn,linenumbers,trackchanges]{aastex631}
%%
%% AASTeX v6.* now includes \hyperref support. While we have built in specific
%% defaults into the classfile you can manually override them with the
%% \hypersetup command. For example,
%%
%% \hypersetup{linkcolor=red,citecolor=green,filecolor=cyan,urlcolor=magenta}
%%
%% will change the color of the internal links to red, the links to the
%% bibliography to green, the file links to cyan, and the external links to
%% magenta. Additional information on \hyperref options can be found here:
%% https://www.tug.org/applications/hyperref/manual.html#x1-40003
%%
%% Note that in v6.3 "bookmarks" has been changed to "true" in hyperref
%% to improve the accessibility of the compiled pdf file.
%%
%% If you want to create your own macros, you can do so
%% using \newcommand. Your macros should appear before
%% the \begin{document} command.
%%
\newcommand{\vdag}{(v)^\dagger}
\newcommand\aastex{AAS\TeX}
\newcommand\latex{La\TeX}

\usepackage{color}
%% Reintroduced the \received and \accepted commands from AASTeX v5.2
%\received{March 1, 2021}
%\revised{April 1, 2021}
%\accepted{\today}

%% Command to document which AAS Journal the manuscript was submitted to.
%% Adds "Submitted to " the argument.
%\submitjournal{PSJ}

%% For manuscript that include authors in collaborations, AASTeX v6.31
%% builds on the \collaboration command to allow greater freedom to 
%% keep the traditional author+affiliation information but only show
%% subsets. The \collaboration command now must appear AFTER the group
%% of authors in the collaboration and it takes TWO arguments. The last
%% is still the collaboration identifier. The text given in this
%% argument is what will be shown in the manuscript. The first argument
%% is the number of author above the \collaboration command to show with
%% the collaboration text. If there are authors that are not part of any
%% collaboration the \nocollaboration command is used. This command takes
%% one argument which is also the number of authors above to show. A
%% dashed line is shown to indicate no collaboration. This example manuscript
%% shows how these commands work to display specific set of authors 
%% on the front page.
%%
%% For manuscript without any need to use \collaboration the 
%% \AuthorCollaborationLimit command from v6.2 can still be used to 
%% show a subset of authors.
%
%\AuthorCollaborationLimit=2
%
%% will only show Schwarz & Muench on the front page of the manuscript
%% (assuming the \collaboration and \nocollaboration commands are
%% commented out).
%%
%% Note that all of the author will be shown in the published article.
%% This feature is meant to be used prior to acceptance to make the
%% front end of a long author article more manageable. Please do not use
%% this functionality for manuscripts with less than 20 authors. Conversely,
%% please do use this when the number of authors exceeds 40.
%%
%% Use \allauthors at the manuscript end to show the full author list.
%% This command should only be used with \AuthorCollaborationLimit is used.

%% The following command can be used to set the latex table counters.  It
%% is needed in this document because it uses a mix of latex tabular and
%% AASTeX deluxetables.  In general it should not be needed.
%\setcounter{table}{1}

%%%%%%%%%%%%%%%%%%%%%%%%%%%%%%%%%%%%%%%%%%%%%%%%%%%%%%%%%%%%%%%%%%%%%%%%%%%%%%%%
%%
%% The following section outlines numerous optional output that
%% can be displayed in the front matter or as running meta-data.
%%
%% If you wish, you may supply running head information, although
%% this information may be modified by the editorial offices.
\shorttitle{late-time spectra of SNe Ia}
\shortauthors{Liu et al.}
%%
%% You can add a light gray and diagonal water-mark to the first page 
%% with this command:
%% \watermark{text}
%% where "text", e.g. DRAFT, is the text to appear.  If the text is 
%% long you can control the water-mark size with:
%% \setwatermarkfontsize{dimension}
%% where dimension is any recognized LaTeX dimension, e.g. pt, in, etc.
%%
%%%%%%%%%%%%%%%%%%%%%%%%%%%%%%%%%%%%%%%%%%%%%%%%%%%%%%%%%%%%%%%%%%%%%%%%%%%%%%%%
\graphicspath{{./}{figures/}}
%% This is the end of the preamble.  Indicate the beginning of the
%% manuscript itself with \begin{document}.

\begin{document}


\title{Implications for explosion mechanism of SNe Ia from late-time spectra(\textcolor{red}{a crude and temporary title})}

%% LaTeX will automatically break titles if they run longer than
%% one line. However, you may use \\ to force a line break if
%% you desire. In v6.31 you can include a footnote in the title.

%% A significant change from earlier AASTEX versions is in the structure for 
%% calling author and affiliations. The change was necessary to implement 
%% auto-indexing of affiliations which prior was a manual process that could 
%% easily be tedious in large author manuscripts.
%%
%% The \author command is the same as before except it now takes an optional
%% argument which is the 16 digit ORCID. The syntax is:
%% \author[xxxx-xxxx-xxxx-xxxx]{Author Name}
%%
%% This will hyperlink the author name to the author's ORCID page. Note that
%% during compilation, LaTeX will do some limited checking of the format of
%% the ID to make sure it is valid. If the "orcid-ID.png" image file is 
%% present or in the LaTeX pathway, the OrcID icon will appear next to
%% the authors name.
%%
%% Use \affiliation for affiliation information. The old \affil is now aliased
%% to \affiliation. AASTeX v6.31 will automatically index these in the header.
%% When a duplicate is found its index will be the same as its previous entry.
%%
%% Note that \altaffilmark and \altaffiltext have been removed and thus 
%% can not be used to document secondary affiliations. If they are used latex
%% will issue a specific error message and quit. Please use multiple 
%% \affiliation calls for to document more than one affiliation.
%%
%% The new \altaffiliation can be used to indicate some secondary information
%% such as fellowships. This command produces a non-numeric footnote that is
%% set away from the numeric \affiliation footnotes.  NOTE that if an
%% \altaffiliation command is used it must come BEFORE the \affiliation call,
%% right after the \author command, in order to place the footnotes in
%% the proper location.
%%
%% Use \email to set provide email addresses. Each \email will appear on its
%% own line so you can put multiple email address in one \email call. A new
%% \correspondingauthor command is available in V6.31 to identify the
%% corresponding author of the manuscript. It is the author's responsibility
%% to make sure this name is also in the author list.
%%
%% While authors can be grouped inside the same \author and \affiliation
%% commands it is better to have a single author for each. This allows for
%% one to exploit all the new benefits and should make book-keeping easier.
%%
%% If done correctly the peer review system will be able to
%% automatically put the author and affiliation information from the manuscript
%% and save the corresponding author the trouble of entering it by hand.

%\correspondingauthor{August Muench}
%\email{greg.schwarz@aas.org, gus.muench@aas.org}

\author{JIALIAN LIU}
\affiliation{Physics Department, Tsinghua University, Beijing 100084, China}
\email{jl-liu18@mails.tsinghua.edu.cn}


%% Note that the \and command from previous versions of AASTeX is now
%% depreciated in this version as it is no longer necessary. AASTeX 
%% automatically takes care of all commas and "and"s between authors names.

%% AASTeX 6.31 has the new \collaboration and \nocollaboration commands to
%% provide the collaboration status of a group of authors. These commands 
%% can be used either before or after the list of corresponding authors. The
%% argument for \collaboration is the collaboration identifier. Authors are
%% encouraged to surround collaboration identifiers with ()s. The 
%% \nocollaboration command takes no argument and exists to indicate that
%% the nearby authors are not part of surrounding collaborations.

%% Mark off the abstract in the ``abstract'' environment. 
\begin{abstract}

The late-time spectra of Type Ia Supernovae (SNe Ia) are important in studying the physics of their explosions. We have collected published late-time optical spectra of 32 SNe Ia at $\sim$ 200 - 400 days after peak magnitude. We also publish two new such late-time spectra of SN 2021hpr in this study. At this phase, the outer ejector has become transparent and the features of inner iron-group elements can be found in the spectra. We use multi-component Gaussian fits to measured the widths, velocity shifts and strengths of iron and nickel features. Then we estimate the nebular velocities and Ni/Fe ratios. We find that the number of SNe Ia which have a redshifted nebular velocity or a blueshifted nebular velocity is similar in our sample. Besides, the majority of SNe Ia are in favor of Sub-M$_{ch}$ models. We also try to find some connections between these late-time parameters and the early-time observations. The correlations imply that the observed velocity shift of Si II 6355 $\rm \AA$ at maximum light increases with higher metallicity and is affected by the viewing angle due to asymmetric explosions.   
\end{abstract}

%% Keywords should appear after the \end{abstract} command. 
%% The AAS Journals now uses Unified Astronomy Thesaurus concepts:
%% https://astrothesaurus.org
%% You will be asked to selected these concepts during the submission process
%% but this old "keyword" functionality is maintained in case authors want
%% to include these concepts in their preprints.
\keywords{supernovae --- general-methods --- statistical}

%% From the front matter, we move on to the body of the paper.
%% Sections are demarcated by \section and \subsection, respectively.
%% Observe the use of the LaTeX \label
%% command after the \subsection to give a symbolic KEY to the
%% subsection for cross-referencing in a \ref command.
%% You can use LaTeX's \ref and \label commands to keep track of
%% cross-references to sections, equations, tables, and figures.
%% That way, if you change the order of any elements, LaTeX will
%% automatically renumber them.
%%
%% We recommend that authors also use the natbib \citep
%% and \citet commands to identify citations.  The citations are
%% tied to the reference list via symbolic KEYs. The KEY corresponds
%% to the KEY in the \bibitem in the reference list below. 

\section{Introduction} \label{sec:intro}

It is widely accepted that type Ia supernovae (SNe Ia; see, e.g. \citealt{1997ARAA..35..309F} for a review of supernova classification) result from thermonuclear explosion of a carbon-oxygen (CO) white dwarf \citep{1997Sci...276.1378N,2000ARAA..38..191H,2014ARAA..52..107M}. Proposed explosion scenarios are generally split into two progenitor systems: the single-degenerate scenario \citep{1973ApJ...186.1007W} with accretion-induced explosion of a massive WD with a non-degenerate companion, and the double-degenerate \citep{1984ApJS...54..335I,1984ApJ...277..355W} scenario with merger-induced explosion of two white dwarfs. Numerical explosion models for explosion scenarios were proposed, such as the M$_{ch}$ W7 models of \citet{1997NuPhA.621..467N} and \citet{1999ApJS..125..439I}, the M$_{ch}$ DDT models of \citet{2013MNRAS.429.1156S}, and the sub-M$_{ch}$ detonation models of \citet{2010ApJ...714L..52S} and \citet{2018ApJ...854...52S}. However, none of the models entirely succeed to be consistent with observations. We still do not understand the explosion mechanism well.

Spectra of SNe Ia are important in studying the explosion mechanism. Many useful parameters can be obtained from the early-time spectra. The postmaximum decline rate $\rm {\Delta}m_{15}(B)$, the decline in magnitudes in the B band during the first 15 days postmaximum, has a correlation with peak luminosity that brighter objects have a lower $\rm {\Delta}m_{15}(B)$ than the dimmer ones (Phillips 1993; Phillips et al. 1999). The velocity shift of Si II 6355 $\rm \AA$ absorption can be used to deduce the photospheric velocity, which gives a direct indication of the kinetic energy of the explosion. However, the outer ejecta is opaque at early time and the information of the inner ejecta is hidden. At phase of about more than 200 days after peak magnitude, the outer ejecta become transparent and the inner region dominated by iron-group elements is visible. Figure \ref{fig:spectrum} shows an example of such late-time spectra. The 7300 $\rm \AA$ region, dominated by [Fe II] and [Ni II] features (e.g., \citealt{2018MNRAS.477.3567M,2020MNRAS.491.2902F}), is important to measured the nebular velocity which can be used to study explosion asymmetry (e.g., \citealt{2010ApJ...708.1703M,2013MNRAS.430.1030S,2018MNRAS.477.3567M}). Besides, Ni/Fe raio inferred from the 7300 $\rm \AA$ region is used to constrain the explosion models in recent studies (e.g., \citealt{2018MNRAS.477.3567M,2020MNRAS.491.2902F,2022arXiv220107864G}), since sub-M$_{ch}$ models predict a lower Ni/Fe ratio compared with M$_{ch}$ models unless the metallicity of the progenitor is high enough. The connection between the Si II velocities (or velocity gradients) at maximum light and the nebular velocities of SNe Ia has been discussed in many works (e.g., \citealt{2010ApJ...708.1703M}). However, other connections between observations near maximum light and late-time observations seem to be rarely mentioned. 

\begin{figure}[ht!]
\gridline{\fig{features.pdf}{0.5\textwidth}{}
		 }  
\caption{Spectrum of SN 2011by from 203 d past maximum brightness. Some prominent features are labelled by shaded areas.} 
\label{fig:spectrum}{}
\end{figure}

This work mainly attempts to study the potential correlations between early-time observations (the Si II velocity at maximum light and $\rm {\Delta}m_{15}(B)$) and late-time observations (the nebular velocity and Ni/Fe ratio). We outline our data sources in Section \ref{sec:source}. In Section \ref{sec:method}, we show the fitting methods and how we roughly estimate the Ni/Fe ratio. Our results are presented in section \ref{sec:results} and the findings are discussed in Section \ref{sec:discussion}. Then we give our conclusions in Section \ref{sec:conclusion}.

\section{Data source} \label{sec:source} 

In order to study the distribution of Ni to Fe ratio for SNe Ia, we collect a sample of 52 late-time spectra ($\sim$200-430 d) of 33 SNe Ia. Peculiar 91T-like and 91bg-like SNe Ia are included except for those exhibiting strong calium features at late time. The publicly available data were retrieved using the Open Supernova Catalog (OSC, \citealt{2017ApJ...835...64G}), Weizmann Interactive Supernova data REPository (WISeREP, \citealt{2012PASP..124..668Y}) and Supernovae Database (SNDB, \citealt{2020MNRAS.492.4325S}), along with spectra of SN 2017fgc \citep{2021ApJ...919...49Z}, SN 2019ein (Xi, in prep), SN 2019np (Sai, in prep) and SN 2021hpr (This work). To study the connection between early-time observations and late-time observations, we have collected Si II velocities at maximum light and post-maximum decline rates $\rm {\Delta}m_{15}(B)$ of these SNe Ia from literatures. To correct the reddening of the spectra, we have also collected redshift from WISeREP and extinction from literatures. All these parameters and references are listed in Table \ref{tab:base_para}. The references of spectra can be found in Table \ref{tab:Multi}. $R_v$ are assumed to be 3.1 except those listed in Table \ref{tab:Rv}. The results should be little affected by the extinction, since we only focus on the 7300 $\rm \AA$ region.

\subsection{SN 2021hpr}
We present 2 new late-time spectra of SN 2021hpr taken with the Lick-3m telescope. An overview of the observations for this supernovae is given in Table \ref{tab:new}.

Lick-3m

\begin{deluxetable*}{cccccccc}
\tablecaption{Overview of the observations.\label{tab:new}}
\tablewidth{0pt}
\tablehead{
\colhead{Name} & \colhead{Observation} & \colhead{Observation} & \colhead{Phase} &
\colhead{$E(B-V)$} & \colhead{redshift} & \colhead{Host galaxy} & \colhead{Exposure}  \\
\colhead{} & \colhead{MJD} & \colhead{date} & \colhead{} &
\colhead{(mag)} & \colhead{} & \colhead{} & \colhead{} 
}
\startdata
SN 2021hpr	 & 59585 & 2022 Jan 6 & +290d & $0.098\pm0.06$ & 0.009346 & NGC 3147 &  
\enddata
\end{deluxetable*}

\section{Method} \label{sec:method}

\subsection{Fitting methods \label{subsec:fitting}}

We focus on the 7300 $\rm \AA$ region which ranges about 6800 - 7800 $\rm \AA$ and is dominated by [Fe II] (7155, 7172, 7388, 7453 $\rm \AA$) and [Ni II] (7378, 7412 $\rm \AA$) \citep{2018MNRAS.477.3567M}. We use a multi-component Gaussian function to fit this region. Here we briefly summarize the processes of the fits. Each spetrum first is corrected for redshift and extinction according to the values presented in Table \ref{tab:base_para}, and then is smoothed with a Savitsky-Golay filter using the scipy package’s $\rm signal.savgol\_filter$ function. The window size of the filter ranges about 30 - 200 $\rm \AA$ depending on the signal-to-noise of the spectra. The fit region should be chosen carefully to obtain a reasonable result. Generally, the local minimum near about 6800 $\rm \AA$ is set to be the blue endpoint. The blue endpoint chosen in this way may induce some flux excess in the blue side of the features, which may be due to [Co III] emissions \citep{2020MNRAS.491.2902F}. Figure \ref{fig:excess} shows the evolution of the flux excess for SN 2011fe, SN2012fr and SN 2014J. It is easy to find that the excess decreases with time and nearly disappears at +350 d, which can be due to the decay of cobalt. The red side of the features near $\sim$ 7600 $\rm \AA$ is usually polluted by absorption features that may come from the earth atmosphere, so we choose a point redder than the absorpton feature as the red endpoint. Distinct excess in the blue side and absorption features in the red side would be removed from the fit region. The pseudo-continuum is defined as a straight line connecting the red and blue endpoints of the fit region. Subtracting the continuum, the $\rm scipy.optimize.curve\_fit$ function is used to fit a multi-component Gaussian function to the features. We follow the fit of \citet{2018MNRAS.477.3567M} that the widths and velocity shifts of the lines of the same element are set to the same value and the relative strengths of the same element are fixed. We use the value of \citet{2015AA...573A..12J}, where [Fe II] 7172, 7388, 7453 $\rm \AA$ features have relative strengths of 0.24, 0.19, 0.31 compared to the [Fe II] 7155 $\rm \AA$ feature and the [Ni II] 7412 $\rm \AA$ feature have a relative strength of 0.31 compared to the [Ni II] 7378 $\rm \AA$ feature. Then there are only 6 free parameters for the multi-component Gaussian function, namely, the widths, velocity shifts and strengths of [Fe II] 7155 $\rm \AA$ and [Ni II] 7378 $\rm \AA$ features. The FWHM of [Ni II] 7378 $\rm \AA$ is limited to be less than 13000$\rm km s^{-1}$, which is adopted by Graham et al. (in prep) to avoid overly-broad-and-shallow nickel features being fit. In practice, we find that overly-broad nickel features would appear when iron features cannot satisfy the region ranges about 7200 - 7300 $\rm \AA$. We can adjust the fit region to avoid overly-broad nickel features in most cases. When the adjustment doesn't work, we try to remove the $\sim$ 7200 - 7300 $\rm \AA$ region. Figure \ref{fig:fit_example0} shows two examples of how we avoid overly-broad nickel features.      

\begin{figure}
\gridline{\fig{4.pdf}{0.5\textwidth}{}
		 }
\caption{Evolutions of flux excess in the blue side of the 7300 $\rm \AA$ region. }
\label{fig:excess}
\end{figure}

\begin{figure*}[ht!]
\gridline{\fig{example21.pdf}{0.5\textwidth}{}
		  \fig{example22.pdf}{0.5\textwidth}{}
		 }
\gridline{\fig{example11.pdf}{0.5\textwidth}{}
		  \fig{example12.pdf}{0.5\textwidth}{}
		 } 
\caption{Some examples for our multi-component Gaussian fits. The reddening corrected observed spectra are shown in gray while the smoothed spectra are shown in black. The overall fits are shown in red, the [Fe II] features are showed in purple dashed lines and the [Ni II] features are showed in green dashed lines. The yellow lines show the region we choose to fit. The continuums are showed in blue. In the top panel, we adjust the fit region to avoid the overly broad nickel features. In the bottom panel, we remove some data points around 7200 $\rm \AA$ to avoid the overly broad nickel features.}
\label{fig:fit_example0}
\end{figure*}

To estimate the uncertainties, we shift the edges of the fit region within 50 $\rm \AA$ (in some low signal-to-noise cases, this is reduced to 10 $\rm \AA$) and re-fit 1000 times. If the flux excess in the red side or the absorption features are removed from the fit region, we also shift the edges of them within 10 $\rm \AA$. A bad fit will be rejected, for example, when the width of Ni is close to 13000 $\rm km s^{-1}$. The standard deviation in the measurements is taken as one of the errors of the Gaussian parameters. Another error comes from the smoothing of the spectrum, which we estimate as the difference between the results given by the smoothed spectrum and the primary spectrum. We also add an uncertainty on the velocity shift measurements of 200 $km\ s{-1}$ to account for peculiar velocity effects of the host galaxies \citep{2018MNRAS.477.3567M}.

\subsection{Rough nickel-to-iron ratio \label{subsec:ni-to-fe}}

The nickel-to-iron ratio is estimated following the method of \citet{2018MNRAS.477.3567M},

\begin{equation}
\frac{n_{Ni\ II}}{n_{Fe\ II}\ }=\frac{L_{7378}}{L_{7155}}\exp{\left(-\frac{0.28}{kT}\right)}\frac{d_{C_{Fe\ II}}}{d_{C_{Ni\ II}}}/4.9
\end{equation}

where $\frac{d_{C_{Fe\ II}}}{d_{C_{Ni\ II}}}$ is the departure coefficients, $\frac{L_{7378}}{L_{7155}}$ is the ratio of luminosity and estimated as the measured flux ratio of [Fe II] 7155 $\rm \AA$ and [Ni II] 7378 $\rm \AA$ features, $k$ is Boltzmann constant and $T$ is the temperature. 

To obtain a rough nickel-to-iron ratio, the departure coefficients $\frac{d_{C_{Fe\ II}}}{d_{C_{Ni\ II}}}$ is assumed to be 1.8, the temperature is assumed to range from 3000K to 8000K and the nickel-to-iron ratio $\frac{n_{Ni}}{n_{Fe}}$ is assumed to equal to $\frac{n_{Ni\ II}\ }{n_{Fe\ II}\ }$. The relative uncertaintiy of the nickel-to-iron ratio estimated in this way is about 40 percent and will be combined with the uncertainty of measured flux ratio to give a total uncertainty.    

\section{Results} \label{sec:results}

The FWMH and velocity shifts of [Fe II] 7155 $\rm \AA$ and [Ni II] 7378 $\rm \AA$ features as well as their flux ratios are listed in Table \ref{tab:Multi}. We use the velocity shifts of [Fe II] 7155 $\rm \AA$ and [Ni II] 7378 $\rm \AA$ features to identify whether the nebular velocities are redshifted or blueshifted, which might have a connection with asymmetric explosion models \citep{2010Natur.466...82M}. We use the flux ratio of [Fe II] 7155 $\rm \AA$ and [Ni II] 7378 $\rm \AA$ features to roughly estimate the Ni/Fe ratio and attempt to confine the explosion models. Inferred nebular velocities and Ni/Fe ratio are also listed in Table \ref{tab:Multi}. Combining the nebular velocities and Ni/Fe ratios with the Si II velocities at maximum light and post maximum decline rates $\rm {\Delta}m_{15}(B)$ presented in Table \ref{tab:base_para}, we find some correlations between these late-time paramaters and early-time parameters.     

\subsection{Nebular velocities \label{subsec:N_v}}

The nebular velocity is estimated as the mean value of velocity shifts of [Fe II] 7155 $\rm \AA$ and [Ni II] 7378 $\rm \AA$ features \citep{2010Natur.466...82M}. When the nebular velocity can be zero within the uncertainty, we do not identify whether it is redshifted or blueshifted and just consider the nebular velocity to be zero. We find 12 SNe Ia have redshifted nebular velocities and 17 SNe Ia have blueshifted nebular velocities in our 33 SNe Ia samples in this way. The two SNe Ia classified by the nebular velocities have a similar number. 

\subsection{Inferred Ni/Fe ratio \label{subsec:ratio}}

Figure \ref{fig:r_p} shows our roughly estimated mass ratio of Ni and Fe as a function of phase after maximum light. Shading regions mark the ranges of predictions for Ni/Fe ratio of different models and we find that the majority (about 2/3) of our SNe Ia samples are in agreement with a double detonation sub-M$_{Ch}$ model, wich is also found by \citet{2020MNRAS.491.2902F}. The left SNe Ia are in agreement with a DDT model or have a critical Ni/Fe ratio between DDT models and sub-$\rm M_{Ch}$ models, except SN 2015F and SN 2002bo whose inferred Ni/Fe ratio may be too high. The reason of the high Ni/Fe ratio of SN 2015F and SN 2002bo is unclear and these two SNe Ia are removed from the following analysis.    

\begin{figure*}[ht!]
\gridline{\fig{ratio_phase.pdf}{1\textwidth}{}
		 }  
\caption{Inferred mass ratio of Ni and Fe as a function of phase after maximum light for SNe Ia collected in this work. Following \citet{2020MNRAS.491.2902F}, the prediction for the Ni/Fe ratio of DDT models locate in the yellow region \citep{2013MNRAS.429.1156S} and that of sub-$\rm M_{Ch}$ models locate in the gray region \citep{2018ApJ...854...52S}.} 
\label{fig:r_p}{}
\end{figure*}

\subsection{Connection between the Ni-to-Fe ratio and the Si II velocity \label{subsec:ratio_v}} 

For SNe Ia that have several spectra in this work, wo choose the spectrum that is closest to 300 days after maixmum light to do the analysis. Figure \ref{fig:R_Si} shows the connection between the Ni/Fe ratio and the velocity of Si II 6355 $\rm \AA$ at maximum light. We find that the Ni/Fe ratio increases with a higher Si II velocity at maximum for both the SNe Ia that have redshifted or blueshifted nebular velocities. The Si II velocity is lower than 12000 $\rm km s^{-1}$ for the SNe Ia that have a blueshifted nebular velocity, while the Si II velocity can approach 16000 $\rm km s^{-1}$ for the SNe Ia that have a redshifted nebular velocity. The range of Ni/Fe ratio is similar for these two SNe Ia. When the Si II velocity is low, about 10000 $\rm km s^{-1}$, the Ni/Fe ratio is similar for these two SNe Ia.   

\begin{figure}[ht!]
\gridline{\fig{ratio_Si_my.pdf}{0.5\textwidth}{}
		 } 
\caption{Inferred mass ratio of Ni and Fe vs. the Si II 6355 $\rm \AA$ velocity at maximum. SNe Ia that have a redshifted (blueshifted) nebular velocity are plot iin red(blue). The SNe Ia plotted in gray have a zero nebular velocity under the uncertainty. The dashed lines are linear best-fits to the corresponding color points.}
\label{fig:R_Si}
\end{figure}

\subsection{Connection between the Ni/Fe ratio and $\rm {\Delta}m_{15}(B)$ \label{subsec:ratio_15}}

Figure \ref{fig:R_15} shows a comparison between the Ni/Fe ratio and $\rm {\Delta}m_{15}(B)$. Different from Figure \ref{fig:R_Si}, red points and blue points are fully mixed. The Ni/Fe ratio increases with $\rm {\Delta}m_{15}(B)$ until $\rm {\Delta}m_{15}(B)$ become very large. When $\rm {\Delta}m_{15}(B)$ is very large, the Ni/Fe ratio become small and the corresponding point locates in the bottom right of the figure. However, where the transformation occurs is unclear, since we lack late-time spectra of SNe Ia that have a $\rm {\Delta}m_{15}(B)$ around 1.6 magnitude in this work.    

\begin{figure}[ht!]
\gridline{\fig{ratio_15_my.pdf}{0.5\textwidth}{}
		 } 
\caption{Inferred mass ratio of Ni and Fe vs. the decline rates $\rm {\Delta}m_{15}(B)$. SNe Ia that have a redshifted (blueshifted) nebular velocity are plot iin red(blue). The SNe Ia plotted in gray have a zero nebular velocity under the uncertainty.}
\label{fig:R_15}
\end{figure}

\section{Discussions} \label{sec:discussion}

We have used multi-component Gaussian fits to estimate the velocity shifts and flux ratios of late-time spectral features in the 7300 $\rm \AA$ region for 33 SNe Ia samples in this work. Nebular velocities and Ni/Fe ratios are inferred from the velocity shifts and the flux ratios respectively. Some relations are found between these inferred late-time parameters and the early-time parameters (Si II velocities at maximum light and $\rm {\Delta}m_{15}(B)$) in section \ref{sec:results}. In this section, we discuss the implications for explosion mechanism from these relations. We also briefly discuss the possible contribution of Calcium in the 7300 $\rm \AA$ region. 

\subsection{Implications for explosion mechanism \label{subsec:implications}} 

\citet{2010Natur.466...82M} find that the SNe Ia having high Si II velocity gradients ($\rm \gtrsim 70 km\ s^{-1}$) display redshifted nebular velocities and propose an asymmetric and off-center explosion toy model to intepret this connection as geometric effect. \citet{2013MNRAS.430.1030S} extend this connection and showed that velocity gradients can be replaced by Si II velocities at maximum light, namely, SNe Ia having high Si II velocities at maximum ($\rm \gtrsim 12000 km\ s^{-1}$) display redshifted nebular velocities. Figure \ref{fig:R_Si} shows that all the SNe Ia that have a high Si II velocity display a redshifted nebular velocity, which is in agreement with \citet{2013MNRAS.430.1030S}.

However, \citet{2013Sci...340..170W} find that the SNe Ia with higher Si II velocity tend to occur nearer the galaxy centers where the stellar metallicity is higher. \citet{2013Sci...340..170W} argue that the diversity of Si II velocities for SNe Ia cannot be completely attributed to geometric effect and the higher Si II velocity may be related in part to a higher metallicity environment. Our work also supports the findings of \citet{2013Sci...340..170W}. \citet{2003ApJ...590L..83T} describe how the ratio of stable-to-radioactive nucleosynthetic products increases with a higher-metallicity progenitor. A higher ratio of stable-to-radioactive nucleosynthetic products means a higher late-time Ni/Fe ratio since all the radioactive $^{56}$Ni would have decayed to $^{56}$Co and finally $^{56}$Fe. If we only focus on the SNe Ia that have a redshifted nebular velocity or that have a blueshifted nebular velocity, we find the Ni/Fe ratio increases with higher Si II velocity, namely, the metallicity of the progenitor has a positive correlation with the Si II velocity. This correlation is also in agreement with \citet{2000ApJ...530..966L} that the blueward shift of the Si II feature increased with higher metallicity.   

In our perspective, metallicity is one of the intrinsic sources of the diversity of Si II velocities while geometric effect caused by asymmetric explosion models affects the observed Si II velocity for the reasons that (1)Figure \ref{fig:R_Si} shows two branches composed of red points (redshifted nebular velocity) and blue points (blueshifted nebular velocity) respectively. The numbers of red points and blue points are close, which implies that we may have equal possibility to find a SNe Ia with redshifted or blueshifted nebular velocity; (2)About half red points have a normal Si II velocity ($\lesssim 12000 km s^{-1}$), which means SNe Ia displaying redshifted nebular velocity is not bound to a high Si II velocity; (3)A positive correlation between the Ni/Fe ratio and the Si II velocity is found for both the red points and blue points. If the viewing angles of all SNe Ia are the same value, we would expect that red points and blue points of Figure \ref{fig:R_Si} were fully mixed just like Figure \ref{fig:R_15}. We also notice that red points and blue point of Figure \ref{fig:R_Si} tend to mix with each other when the Si II velocity is about 10000 $km s^{-1}$. For these SNe Ia, symmetric explosion models may be more suitable.

The correlation between the Ni/Fe ratio and $\rm {\Delta}m_{15}(B)$ showed in Figure \ref{fig:R_15} implies that the inferred Ni/Fe ratio may be an important parameter for SNe Ia just like $\rm {\Delta}m_{15}(B)$. This correlation is reasonalble. A higher Ni/Fe ratio means a higher ratio of stable-to-radioactive nucleosynthetic products. The radioactive $^{56}$Ni is an important energy source for the luminosity of SNe Ia at early time, so the potmaximum decline rate $\rm {\Delta}m_{15}(B)$ increases with a higher of stable-to-radioactive nucleosynthetic products if the total quantity of nickel is similar. Although exceptions appear in the bottom right of Figure \ref{fig:R_15} that have a large $\rm {\Delta}m_{15}(B)$ and a low Ni/Fe ratio, they are 91bg-like SNe Ia which have less $^{56}$Ni compared to the normal SNe Ia and most are found in old stellar population \citep{2019PASA...36...31P} where the metallicity is low.

\subsection{Calcium in the 7300 $\rm \AA$ region \label{subsec:Calcium}}

Whether calcium has a contribution to the 7300 $\rm \AA$ region of the late-time SNe Ia spectrum has been discussed in many studies. \citet{2018MNRAS.477.3567M} use the Bayesian Information Criterion to select between models with and without a contribution from calcium and the results are in favour of the no calcium models for their SNe Ia samples. \citet{2020MNRAS.491.2902F} use the 19390 $\rm \AA$ line to constrain the 7378 $\rm \AA$ line for high SNR spectra and find the contribution of calcium is very limited in the 7300 $\rm \AA$ region. However, \citet{2021arXiv211100016T} identify [Ca II] 7291 $\rm \AA$, 7324 $\rm \AA$ doublet in the spectrum of SN 2011fe at about 480 days, where a central emission peak appears and the emission profile transitions from double- to triple-peaked (see Figure 1 in \citealt{2021arXiv211100016T}). And the nickel features measured from our multi-component Gaussian fits, where possible contribution of calcium is neglected, have overly-broad widths and much more blueward shifts compared with the iron features in some cases. If the calcium is not negligible, this unexpected results would be understandable since the nickel features given by our fits would be the blended emission lines of nickel and calcium.   

We have attempted to reduce the possible effect of calcium on our fits. We try to remove some data points of the spectrum before fitting when overly-broad nickel features appear. This is a crude but effective method in some cases (see the bottom panel of Figure \ref{fig:fit_example0}).  Nevertheless, there are only one spectrum which we have to use this method to avoid overly-broad nickel features in this work and our method is effective on it. Thus, the correlations found in section \ref{sec:results} are little affected. Note that we also remove some data points around 7300 $\rm \AA$ of the spectra of SN 2017fgc and SN 2019ein due to the unexpected narrow emission whose source is unclear.       

\section{Conclusions} \label{sec:conclusion}

We have performed multi-component Gaussian fits to the 7300 $\rm \AA$ [Fe II]/[Ni II]-dominated region for 33 SNe Ia collected in this work. This has allowed us to measured the velocity shifts and flux ratios of [Ni II] and [Fe II] features in this region, which can be used to inferred Nebular velocities and late-time Ni/Fe ratios respectively. Connecting these inferred late-time parameters with the early-time observations (Si II velocity at maximum light and $\rm {\Delta}m_{15}(B)$), we find some interesting connections between them.

Our main results are:

(i)The majority (about 2/3) of SNe Ia in this work have a Ni/Fe ratio in the range of 0.02 - 0.06 with a uncertainty of more than 40 percent, in favour of sub-$\rm M_{Ch}$ models.   

(ii)Metallicity is one of the intrinsic sources of the diversity of Si II velocities while geometric effect caused by asymmetric explosion models affects the observed Si II velocity. For the SNe Ia that have a Si II velocity about 10000 $km s^{-1}$, symmetric explosion models may be more suitable. 

(iii)The Ni/Fe ratio has a positive correlation with $\rm {\Delta}m_{15}(B)$ except for 91bg-like SNe Ia which have a large $\rm {\Delta}m_{15}(B)$ and low metallicity environment. 

(iv)Although a possible contribution of calcium cannot be completely ruled out, our results are little affected.

Connecting the early-time and late-time observations of SNe Ia allows us to understand the explosion mechanism better. In future work, more high-quality late-time spectra are needed to confirm these connections and find some new connections. 

\begin{acknowledgments}
\textcolor{red}{to be written}
\end{acknowledgments}

\bibliography{paper}{}
\bibliographystyle{aasjournal}

\appendix

\section{OVERVIEW OF SNe Ia AND FITTING RESULTS} \label{sec:spectra}

Table \ref{tab:base_para} presents basic information of SNe Ia in this work. Table \ref{tab:Rv} presents the SNe Ia whose $\rm R_v$ is not assumed to be 3.1. Figure \ref{fig:all_fit} shows the fitting results for all spectra in this work and the corresponding parameters are presented in Table \ref{tab:Multi}.  

\startlongtable
\begin{deluxetable*}{ccccccccccc}
\tablenum{A1}
\tablecaption{SNe Ia light curve and spectral parameters, host galaxy information, number of late-time spectra and the corresponding phases.\label{tab:base_para}}
\tablewidth{0pt}
\tablehead{
\colhead{Name} & \colhead{Host galaxy} & \colhead{Redshift} & \colhead{E(B-V)} &
\colhead{$\rm {\Delta}m15(B)$} & \colhead{$\rm v_0(Si\ II)$} & \colhead{$N_{spec}$} &
\colhead{Phase} & \colhead{Ref.} & \colhead{Ref.} & \colhead{Ref.} \\
\colhead{} & \colhead{} & \colhead{} & \colhead{(mag)} &
\colhead{(mag)} & \colhead{1000 $\rm km\ s^{-1}$} & \colhead{} &
\colhead{d} & \colhead{$\rm v_0(Si\ II)$} & \colhead{$\rm {\Delta}m15(B)$} & \colhead{E(B-V)}
}
\startdata
SN1986G	    &   NGC 5128  &	0.001825 &	1.1    & 	1.81$\pm$0.07 &	10.00$\pm$0.15 &	1 &	256	     &  1  &	2  &	3 \\
SN1990N	    &   NGC 4639  & 0.003369 &	0.02   &	0.95$\pm$0.05 &	10.53$\pm$0.15 &	3 &	227-305  &	1  &	4  &	5,6 \\
SN1998bu    &	NGC 3368  &	0.002992 &	0.34   &	1.06$\pm$0.04 &	10.50$\pm$0.10 &	2 &	237, 281 &	7  &	7  &	9 \\
SN1999aa    &	NGC 2595  &	0.014907 &	0.03   & 	0.81$\pm$0.02 & 10.50$\pm$0.20 &	1 &	256      &	7  & 	7  &	5,6 \\
SN2002bo    &	NGC 3190  &	0.0043   &	0.43   &	1.15$\pm$0.03 &	13.20$\pm$0.20 &	1 &	311      &	7  & 	7  &	9 \\
SN2003du    &	UGC 9391  &	0.006408 &	0.01   &	1.00$\pm$0.02 &	10.40$\pm$0.30 &	1 &	219      &	7  &	7  &	10 \\
SN2003gs    &	NGC 936	  & 0.00477	 &  0.07   &	1.93$\pm$0.07 &	11.40$\pm$0.30 &	1 &	207	     &  7  &	7  &	11 \\
SN2003hv    &	NGC 1201  &	0.005624 &	-0.03  &	1.45$\pm$0.07 &	11.30$\pm$0.30 &	1 &	319	     &  7  &	7  &	2,6,12 \\
SN2003kf    &	PGC 18373 &	0.00739	 &  0.27   &    0.93$\pm$0.04 & 11.10$\pm$0.30 &	1 &	400	     &  7  &	7  &	5,6 \\
SN2004eo    &	NGC 6928  &	0.015718 &	0.23   &	1.40$\pm$0.03 &	10.70$\pm$0.30 &	1 &	227	     &  7  &    7  &	5,6,13 \\
SN2006X	    &   NGC 4321  &	0.005294 &	1.42   &	1.26$\pm$0.05 &	16.10$\pm$0.20 &	2 &	276, 359 &	7  &	7  &	14 \\
SN2007af    &	NGC 5584  &	0.005464 &	0.21   &	1.16$\pm$0.03 &	10.80$\pm$0.20 &	1 &	303	     &  7  &	7  &	2,6,13 \\
SN2007le    &	NGC 7721  &	0.006721 &	0.37   &	1.02$\pm$0.05 &	12.90$\pm$0.60 &	1 &	306	     &  7  &	7  &	15 \\
SN2008Q	    &   NGC 524	  & 0.0081	 &  0.07   &	1.25$\pm$0.08 &	11.09$\pm$0.10 &	1 &	200	     &  16 &	16 &	5,6 \\
SN2011by    &	NGC 3972  &	0.002843 &	0.01   &	1.14$\pm$0.03 &	10.35$\pm$0.14 &	2 &	207, 311 & 	16 &	16 &	5,6 \\
SN2011fe    &	NGC 5457  & 0.000804 &	0.01   &	1.18$\pm$0.03 &	10.40$\pm$0.20 &	5 &	203-378  &	7  &	7  &	5,6 \\
SN2012cg    &	NGC 4424  &	0.001458 &	0.20   &	0.83$\pm$0.03 &	10.00$\pm$0.20 &	2 &	286, 342 &	17 &	4  &	5,6,18 \\
SN2012fr    &	NGC 1365  &	0.004	 &  0.03   &    0.85$\pm$0.05 & 12.00$\pm$0.20 &	4 &	222-367  &  17 &	7  &	5,6 \\
SN2012hr    &	PGC 18880 &	0.008	 &  0.04   &	1.04$\pm$0.01 &	11.50$\pm$0.20 &	1 &	284	     &  19 &	19 &	5,6 \\
SN2013aa    &	NGC 5643  &	0.003999 &	0.21   &	0.95$\pm$0.01 &	10.20$\pm$0.20 &	1 &	400      &  19 &	19 &	5,6 \\
SN2013cs    &	ESO576-17 &	0.00924	 &  0.08   &	0.81$\pm$0.18 &	12.50$\pm$0.20 &	2 &	261, 303 &	19 &	19 &	5,6 \\
SN2013dy    &	NGC 7250  &	0.00389	 &  0.28   &	0.92$\pm$0.01 &	10.30$\pm$0.20 &	1 &	333,	 &  20 &	4  &	5,6,21 \\
SN2013gy    &	NGC 1418  &	0.014023 &	0.155  &	1.20$\pm$0.03 &	10.70$\pm$0.20 &	1 &	275    	 &  19 &	4  &	5,6,22 \\
SN2014J	    &   NGC 3034  & 0.000677 &	1.19   &	0.98$\pm$0.02 &	12.10$\pm$0.20 &	3 &	263-349	 &  7  &	7  &	23 \\
SN2015F	    &   NGC 2422  & 0.0049	 &  0.21   &	1.18$\pm$0.02 &	10.10$\pm$0.20 &	1 &	295	     &  19 &	19 &	5,6,24 \\
SN2017cbv   &	NGC 5643  &	0.003999 &	0.16   &	0.99$\pm$0.01 &	9.30$\pm$0.06  &	1 &	318 	 &  25 &	26 &	5,6 \\
SN2017fgc   &	NGC 0474  &	0.001458 &	0.20   &	1.05$\pm$0.07 &	15.20$\pm$0.20 &	1 &	384      &	27 &	27 &	5,6,27 \\
SN2018oh    &	UGC 04780 &	0.012	 &  0.04   &    0.96$\pm$0.03 & 10.10$\pm$0.10 &	1 &	259	     &  28 &	28 &	5,6 \\
SN2019ein   &	NGC 5353  &	0.007755 &	0.1    &	1.35$\pm$0.01 &	14.00$\pm$0.20 &	1 &	311 	 &  29 &    29 &	5,6,30 \\
SN2019np    &	NGC 3254  &	0.00452	 &  0.12   &    1.04$\pm$0.04 & 10.00$\pm$0.10 &	2 &	303, 368 &	31 &    31 &	5,6,31 \\
SN2021hpr   &   NGC 3147  & 0.009346 &  0.1    &    1.08$\pm$0.02 & 11.09$\pm$0.08 &	1 &	250      &	?  &    ?  &	? \\
ASASSN-14jg &   PGC128348 & 0.014827 &  0.01   &    0.92$\pm$0.01 & 10.30$\pm$0.20 &    3 &	222-325  &  19 &    19 &	5,6 \\
\enddata
\tablecomments{Reference: (1) \citet{2011ApJ...729...55F}; (2) \citet{1992AA...259...63C}; (3) \citet{1987PASP...99..592P}; (4) \citet{2015ApJS..220...20Z}; (5) \citet{2011ApJ...737..103S}; (6) \citet{2017ApJ...835...64G}; (7) \citet{2019ApJ...882..120W}; (8) \citet{1999AJ....117.1175S}; (9) \citet{2004MNRAS.348..261B}; (10) \citet{2005AA...429..667A} (11) \citet{2009AJ....138.1584K} (12) \citet{2009AA...505..265L} (13) \citet{2014ApJ...789...32B}; (14) \citet{2008ApJ...675..626W}; (15) \citet{2013ApJ...779...38P}; (16) \citet{2013MNRAS.430.1030S}; (17) \citet{2018MNRAS.477.3567M}; (18) \citet{2012ApJ...756L...7S}; (19) \citet{2017MNRAS.472.3437G}; (20) \citet{2015MNRAS.452.4307P}; (21) \citet{2013ApJ...778L..15Z}; (22) \citet{2019AA...627A.174H} (23) \citet{2014MNRAS.443.2887F} (24) \citet{2015ApJS..221...22I} (25) Grahem et al, in prep; (26) \citet{2020ApJ...904...14W}; (27) \citet{2021ApJ...919...49Z} (28) \citet{2019ApJ...870...12L} (29) Xi et al, in prep (30) \citet{2020ApJ...893..143K} (31) Sai et al, in prep.}
\end{deluxetable*}

\startlongtable
\begin{deluxetable*}{ccl}
\tablenum{A2}
\tablecaption{$R_v$ values and references of SNe Ia whose $R_v$ values are not assumed to be 3.1.\label{tab:Rv}}
\tablewidth{0pt}
\tablehead{
\colhead{Name} & \colhead{$R_v$} & \colhead{Ref.}
}
\startdata
SN2006X	 &  1.5	& \citet{2008ApJ...675..626W}; \citet{2013ApJ...779...38P}; \citet{2014ApJ...789...32B} \\
SN2014J	 &  1.5	& \citet{2014ApJ...788L..21A}; \citet{2014MNRAS.443.2887F}; \citet{2015ApJ...807L..26G}; \citet{2015ApJ...805...74B} \\
SN2017fgc &	1.5	& \citet{2021MNRAS.502.4112B} \\
SN2019ein &	1.5	& \citet{2020ApJ...893..143K} \\
\enddata
\end{deluxetable*}



\startlongtable
\begin{deluxetable*}{cccccccccc}
\tablenum{A3}
\tablecaption{Multi-component Gaussian fits parameters of nebular-phase emission lines and their inferred nebular velocities and Ni/Fe ratios.\label{tab:Multi}}
\tablewidth{0pt}
\tabletypesize{\scriptsize}
\tablehead{
\colhead{Name} & \colhead{Phase} & \colhead{[Fe II] Velocity} & \colhead{[Ni II] Velocity} & 
\colhead{[Fe II] FWHM} & \colhead{[Ni II] FWHM} & \colhead{Flux ratio} & \colhead{Nebular Vel.} & \colhead{Inferred} &  \colhead{Ref.} \\
\colhead{} & \colhead{[days]} & \colhead{[km s$^{-1}$]} & \colhead{[km s$^{-1}$]} & 
\colhead{[km s$^{-1}$]} & \colhead{[km s$^{-1}$]} & \colhead{Ni/Fe}  & \colhead{[km s$^{-1}$]} & \colhead{Ni/Fe} & \colhead{Spec.} 
}
\startdata
SN1986G & 256 & -17$\pm$244 & -1133$\pm$263 & 9614$\pm$158 & 2979$\pm$306 & 0.170$\pm$0.008 & -575$\pm$179 & 0.032$\pm$0.013 & 1 \\ 
SN1990N & 227 & -1691$\pm$241 & -3581$\pm$427 & 8933$\pm$213 & 6506$\pm$149 & 0.108$\pm$0.008 & -2636$\pm$245 & 0.020$\pm$0.008 & 2 \\ 
SN1990N & 280 & -1350$\pm$375 & -1821$\pm$916 & 9330$\pm$390 & 6690$\pm$1307 & 0.158$\pm$0.065 & -1586$\pm$495 & 0.029$\pm$0.017 & 2 \\ 
SN1990N & 305 & -1312$\pm$234 & -1961$\pm$459 & 9075$\pm$131 & 8217$\pm$352 & 0.192$\pm$0.018 & -1637$\pm$257 & 0.036$\pm$0.015 & 2 \\ 
SN1998bu & 237 & -1379$\pm$248 & -1704$\pm$287 & 8007$\pm$241 & 5642$\pm$305 & 0.487$\pm$0.019 & -1542$\pm$190 & 0.090$\pm$0.036 & BSNIP \\ 
SN1998bu & 281 & -1455$\pm$260 & -1732$\pm$308 & 8325$\pm$144 & 4861$\pm$359 & 0.357$\pm$0.011 & -1594$\pm$202 & 0.066$\pm$0.027 & BSNIP \\ 
SN1999aa & 258 & 115$\pm$265 & -209$\pm$279 & 8077$\pm$106 & 3402$\pm$221 & 0.138$\pm$0.010 & -46$\pm$192 & 0.026$\pm$0.010 & BSNIP \\ 
SN2002bo & 311 & 1324$\pm$260 & 1823$\pm$421 & 8063$\pm$1171 & 7194$\pm$229 & 0.854$\pm$0.153 & 1573$\pm$247 & 0.158$\pm$0.069 & 3 \\ 
SN2003du & 219 & -1994$\pm$292 & -3010$\pm$735 & 8159$\pm$356 & 8121$\pm$812 & 0.323$\pm$0.085 & -2502$\pm$395 & 0.060$\pm$0.029 & 6 \\ 
SN2003gs & 207 & 1604$\pm$234 & 1356$\pm$256 & 10954$\pm$243 & 3709$\pm$25 & 0.160$\pm$0.006 & 1480$\pm$173 & 0.030$\pm$0.012 & BSNIP \\ 
SN2003hv & 319 & -2582$\pm$246 & -3716$\pm$315 & 8688$\pm$204 & 5462$\pm$562 & 0.563$\pm$0.052 & -3149$\pm$200 & 0.104$\pm$0.043 & 7 \\ 
SN2003kf & 400 & 304$\pm$396 & 117$\pm$729 & 10732$\pm$214 & 4962$\pm$1596 & 0.144$\pm$0.021 & 210$\pm$415 & 0.027$\pm$0.011 & CfA \\ 
SN2004eo & 227 & -1029$\pm$247 & -2346$\pm$348 & 8363$\pm$149 & 8699$\pm$316 & 0.459$\pm$0.039 & -1688$\pm$213 & 0.085$\pm$0.035 & 8 \\ 
SN2005cf & 317 & 216$\pm$444 & 308$\pm$537 & 9055$\pm$250 & 5808$\pm$274 & 0.165$\pm$0.018 & 262$\pm$348 & 0.031$\pm$0.013 & BSNIP \\ 
SN2006X & 276 & 2379$\pm$273 & 2383$\pm$263 & 8586$\pm$323 & 4833$\pm$60 & 0.398$\pm$0.030 & 2381$\pm$189 & 0.074$\pm$0.030 & BSNIP \\ 
SN2006X & 359 & 2740$\pm$235 & 2496$\pm$245 & 9238$\pm$214 & 5291$\pm$73 & 0.531$\pm$0.024 & 2618$\pm$170 & 0.098$\pm$0.040 & BSNIP \\ 
SN2007af & 303 & 402$\pm$227 & 332$\pm$363 & 7787$\pm$387 & 5748$\pm$247 & 0.254$\pm$0.030 & 367$\pm$214 & 0.047$\pm$0.020 & CfA \\ 
SN2007le & 306 & 1481$\pm$248 & 1599$\pm$297 & 8872$\pm$296 & 5204$\pm$78 & 0.247$\pm$0.019 & 1540$\pm$193 & 0.046$\pm$0.019 & BSNIP \\ 
SN2008Q & 200 & -2730$\pm$309 & -1697$\pm$221 & 10661$\pm$227 & 4183$\pm$82 & 0.394$\pm$0.023 & -2213$\pm$190 & 0.073$\pm$0.030 & BSNIP \\ 
SN2011by & 207 & -1531$\pm$277 & -2830$\pm$485 & 7718$\pm$281 & 5568$\pm$439 & 0.246$\pm$0.029 & -2181$\pm$279 & 0.046$\pm$0.019 & 9 \\ 
SN2011by & 311 & -1064$\pm$212 & -1711$\pm$242 & 8016$\pm$158 & 5099$\pm$40 & 0.318$\pm$0.013 & -1387$\pm$161 & 0.059$\pm$0.024 & 9 \\ 
SN2011fe & 203 & -1141$\pm$210 & -2210$\pm$238 & 8473$\pm$75 & 5629$\pm$54 & 0.222$\pm$0.007 & -1675$\pm$158 & 0.041$\pm$0.016 & 10 \\ 
SN2011fe & 224 & -1026$\pm$229 & -2070$\pm$308 & 8195$\pm$130 & 5732$\pm$146 & 0.254$\pm$0.012 & -1548$\pm$192 & 0.047$\pm$0.019 & 10 \\ 
SN2011fe & 309 & -1066$\pm$250 & -1076$\pm$331 & 8379$\pm$218 & 7106$\pm$225 & 0.395$\pm$0.022 & -1071$\pm$207 & 0.073$\pm$0.030 & 10 \\ 
SN2011fe & 345 & -638$\pm$251 & -1295$\pm$284 & 8558$\pm$109 & 6955$\pm$128 & 0.422$\pm$0.014 & -966$\pm$189 & 0.078$\pm$0.031 & 10 \\ 
SN2011fe & 378 & -340$\pm$296 & -1047$\pm$355 & 9109$\pm$292 & 7014$\pm$225 & 0.397$\pm$0.038 & -693$\pm$231 & 0.074$\pm$0.030 & 10 \\ 
SN2012cg & 286 & -1069$\pm$223 & -1825$\pm$354 & 7857$\pm$169 & 5625$\pm$95 & 0.187$\pm$0.005 & -1447$\pm$209 & 0.035$\pm$0.014 & 11 \\ 
SN2012cg & 342 & -1109$\pm$234 & -840$\pm$315 & 8771$\pm$106 & 7316$\pm$148 & 0.267$\pm$0.010 & -974$\pm$196 & 0.050$\pm$0.020 & 12 \\ 
SN2012fr & 222 & 2047$\pm$265 & 2939$\pm$400 & 8078$\pm$97 & 5672$\pm$159 & 0.144$\pm$0.007 & 2493$\pm$240 & 0.027$\pm$0.011 & 13 \\ 
SN2012fr & 261 & 2052$\pm$230 & 2510$\pm$495 & 8064$\pm$1133 & 5262$\pm$95 & 0.159$\pm$0.012 & 2281$\pm$273 & 0.029$\pm$0.012 & 13 \\ 
SN2012fr & 340 & 2116$\pm$242 & 2686$\pm$236 & 8689$\pm$257 & 5086$\pm$61 & 0.285$\pm$0.013 & 2401$\pm$169 & 0.053$\pm$0.021 & 13 \\ 
SN2012fr & 367 & 2130$\pm$267 & 3263$\pm$252 & 9212$\pm$1022 & 4961$\pm$440 & 0.225$\pm$0.030 & 2697$\pm$184 & 0.042$\pm$0.018 & 13 \\ 
SN2012hr & 284 & 133$\pm$241 & 71$\pm$649 & 8392$\pm$357 & 7594$\pm$891 & 0.168$\pm$0.042 & 102$\pm$346 & 0.031$\pm$0.015 & 13 \\ 
SN2013aa & 400 & -738$\pm$238 & -1150$\pm$262 & 8181$\pm$120 & 6999$\pm$226 & 0.283$\pm$0.014 & -944$\pm$177 & 0.053$\pm$0.021 & 14 \\ 
SN2013cs & 261 & 1477$\pm$234 & 2634$\pm$247 & 9293$\pm$68 & 4854$\pm$74 & 0.147$\pm$0.003 & 2055$\pm$170 & 0.027$\pm$0.011 & 14 \\ 
SN2013cs & 303 & 1352$\pm$247 & 1130$\pm$368 & 7847$\pm$166 & 6717$\pm$165 & 0.254$\pm$0.011 & 1241$\pm$221 & 0.047$\pm$0.019 & 12 \\ 
SN2013dy & 333 & -1169$\pm$224 & -1629$\pm$260 & 7600$\pm$78 & 6787$\pm$229 & 0.223$\pm$0.009 & -1399$\pm$171 & 0.041$\pm$0.017 & 15 \\ 
SN2013gy & 275 & -389$\pm$211 & -533$\pm$427 & 8140$\pm$218 & 7166$\pm$170 & 0.324$\pm$0.013 & -461$\pm$238 & 0.060$\pm$0.024 & 13 \\ 
SN2014J & 263 & 550$\pm$217 & 595$\pm$312 & 8313$\pm$133 & 7224$\pm$253 & 0.241$\pm$0.018 & 572$\pm$190 & 0.045$\pm$0.018 & 10 \\ 
SN2014J & 303 & 664$\pm$222 & 485$\pm$333 & 8072$\pm$135 & 7222$\pm$143 & 0.289$\pm$0.012 & 574$\pm$200 & 0.053$\pm$0.022 & 10 \\ 
SN2014J & 349 & 628$\pm$312 & 929$\pm$312 & 8517$\pm$520 & 8847$\pm$1027 & 0.459$\pm$0.094 & 779$\pm$220 & 0.085$\pm$0.038 & 17 \\ 
SN2015F & 279 & -549$\pm$303 & -1245$\pm$255 & 7304$\pm$157 & 9769$\pm$812 & 0.854$\pm$0.136 & -897$\pm$198 & 0.158$\pm$0.068 & 14 \\ 
SN2017cbv & 318 & -815$\pm$366 & -1714$\pm$443 & 7764$\pm$423 & 7221$\pm$347 & 0.262$\pm$0.027 & -1265$\pm$287 & 0.049$\pm$0.020 & 19 \\ 
SN2017fgc & 384 & 3539$\pm$1287 & 3497$\pm$3423 & 9272$\pm$1682 & 6376$\pm$2340 & 0.362$\pm$0.164 & 3518$\pm$1828 & 0.069$\pm$0.042 & 20 \\ 
SN2018oh & 259 & -1847$\pm$375 & -3910$\pm$804 & 7052$\pm$342 & 5319$\pm$666 & 0.218$\pm$0.047 & -2878$\pm$444 & 0.041$\pm$0.018 & 19 \\ 
SN2019ein & 311 & 1665$\pm$2974 & 2473$\pm$770 & 11151$\pm$6624 & 3745$\pm$1376 & 0.334$\pm$0.191 & 2069$\pm$1536 & 0.064$\pm$0.045 & 21 \\ 
SN2019np & 303 & -1695$\pm$226 & -3240$\pm$410 & 7014$\pm$81 & 5542$\pm$241 & 0.233$\pm$0.008 & -2467$\pm$234 & 0.043$\pm$0.017 & 22 \\ 
SN2019np & 368 & -1522$\pm$317 & -3042$\pm$401 & 7165$\pm$333 & 5662$\pm$77 & 0.305$\pm$0.018 & -2282$\pm$256 & 0.057$\pm$0.023 & 22 \\ 
SN2021hpr & 250 & 852$\pm$229 & 279$\pm$552 & 7734$\pm$933 & 5307$\pm$730 & 0.324$\pm$0.019 & 565$\pm$299 & 0.060$\pm$0.024 & This work \\ 
ASASSN-14jg & 222 & 1335$\pm$235 & 1389$\pm$383 & 8397$\pm$261 & 2763$\pm$289 & 0.050$\pm$0.009 & 1362$\pm$225 & 0.009$\pm$0.004 & 19 \\ 
ASASSN-14jg & 268 & 1609$\pm$235 & 1071$\pm$355 & 7832$\pm$238 & 4207$\pm$339 & 0.076$\pm$0.012 & 1340$\pm$213 & 0.014$\pm$0.006 & 14 \\ 
ASASSN-14jg & 325 & 1769$\pm$255 & 2335$\pm$277 & 8774$\pm$664 & 5010$\pm$93 & 0.149$\pm$0.030 & 2052$\pm$188 & 0.028$\pm$0.012 & 23 \\ 
\enddata
\tablecomments{Reference: (1) \citet{1992AA...259...63C}; (2) \citet{1996AJ....112.2094G}; (3) \citet{2012AJ....143..126B}; (4) \citet{2007AA...469..645S}; (5) \citet{2009AA...505..265L}; (6) \citet{2007MNRAS.376.1301P}; (7) \citet{2013MNRAS.430.1030S}; (8) \citet{2020MNRAS.492.4325S} (9) \citet{2018ApJ...855....6S} (10) \citet{2016MNRAS.457.3254M}; (11) \citet{2015MNRAS.454.3816C}; (12) \citet{2017MNRAS.472.3437G}; (13) \citet{2015MNRAS.452.4307P}; (14) \citet{2016MNRAS.457..525G}; (15) \citet{2016MNRAS.457.1000S}; (16) \citet{2018MNRAS.481..878Z}; (17) \citet{2019ApJ...872L..22T}; (18) \citet{2021ApJ...919...49Z}; (19) Xi et al., in prep; (20) \citet{2018MNRAS.477.3567M}; (21) Xi et al., in prep (22) Sai et al., in prep.}
\end{deluxetable*}

\begin{figure*}
\gridline{\fig{SN1986G_256.pdf}{0.25\textwidth}{}
          \fig{SN1990N_227.pdf}{0.25\textwidth}{}
          \fig{SN1990N_280.pdf}{0.25\textwidth}{}
          \fig{SN1990N_305.pdf}{0.25\textwidth}{}
          }
\gridline{\fig{SN1998bu_237.pdf}{0.25\textwidth}{}
          \fig{SN1998bu_281.pdf}{0.25\textwidth}{}
          \fig{SN1999aa_258.pdf}{0.25\textwidth}{}
          \fig{SN2002bo_311.pdf}{0.25\textwidth}{}
          }
\gridline{\fig{SN2003du_219.pdf}{0.25\textwidth}{}
          \fig{SN2003gs_207.pdf}{0.25\textwidth}{}
          \fig{SN2003hv_319.pdf}{0.25\textwidth}{}
          \fig{SN2003kf_400.pdf}{0.25\textwidth}{}
          }
\gridline{\fig{SN2004eo_227.pdf}{0.25\textwidth}{}
          \fig{SN2005cf_317.pdf}{0.25\textwidth}{}
          \fig{SN2006X_276.pdf}{0.25\textwidth}{}
          \fig{SN2006X_359.pdf}{0.25\textwidth}{}
          }
\gridline{\fig{SN2007af_303.pdf}{0.25\textwidth}{}
          \fig{SN2007le_306.pdf}{0.25\textwidth}{}
          \fig{SN2008Q_200.pdf}{0.25\textwidth}{}
          \fig{SN2011by_207.pdf}{0.25\textwidth}{}
          }
\end{figure*} 
\begin{figure*}
\gridline{\fig{SN2011by_311.pdf}{0.25\textwidth}{}
          \fig{SN2011fe_203.pdf}{0.25\textwidth}{}
          \fig{SN2011fe_224.pdf}{0.25\textwidth}{}
          \fig{SN2011fe_309.pdf}{0.25\textwidth}{}
          }
\gridline{\fig{SN2011fe_345.pdf}{0.25\textwidth}{}
          \fig{SN2011fe_378.pdf}{0.25\textwidth}{}
          \fig{SN2012cg_286.pdf}{0.25\textwidth}{}
          \fig{SN2012cg_342.pdf}{0.25\textwidth}{}
          }
\gridline{\fig{SN2012fr_222.pdf}{0.25\textwidth}{}
          \fig{SN2012fr_261.pdf}{0.25\textwidth}{}
          \fig{SN2012fr_340.pdf}{0.25\textwidth}{}
          \fig{SN2012fr_367.pdf}{0.25\textwidth}{}
          }
\gridline{\fig{SN2012hr_284.pdf}{0.25\textwidth}{}
          \fig{SN2013aa_400.pdf}{0.25\textwidth}{}
          \fig{SN2013cs_261.pdf}{0.25\textwidth}{}
          \fig{SN2013cs_303.pdf}{0.25\textwidth}{}
          }
\gridline{\fig{SN2013dy_333.pdf}{0.25\textwidth}{}
          \fig{SN2013gy_275.pdf}{0.25\textwidth}{}
          \fig{SN2014J_263.pdf}{0.25\textwidth}{}
          \fig{SN2014J_303.pdf}{0.25\textwidth}{}
          }
\end{figure*} 
\begin{figure*}
\figurenum{A1}
\gridline{\fig{SN2014J_349.pdf}{0.25\textwidth}{}
          \fig{SN2015F_279.pdf}{0.25\textwidth}{}
          \fig{SN2017cbv_318.pdf}{0.25\textwidth}{}
          \fig{SN2017fgc_384.pdf}{0.25\textwidth}{}
          }
\gridline{\fig{SN2018oh_259.pdf}{0.25\textwidth}{}
          \fig{SN2019ein_311.pdf}{0.25\textwidth}{}
          \fig{SN2019np_303.pdf}{0.25\textwidth}{}
          \fig{SN2019np_368.pdf}{0.25\textwidth}{}
          }
\gridline{\fig{SN2021hpr_250.pdf}{0.25\textwidth}{}
          \fig{ASASSN-14jg_222.pdf}{0.25\textwidth}{}
          \fig{ASASSN-14jg_268.pdf}{0.25\textwidth}{}
          \fig{ASASSN-14jg_325.pdf}{0.25\textwidth}{}
          } 
\caption{Best fits to the 7300 $\rm \AA$ region for all samples in this work. The reddening corrected observed spectra are shown in gray while the smoothed spectra are shown in black. The overall fits are shown in red, the [Fe II] features are showed in purple dashed lines and the [Ni II] features are showed in green dashed lines. The yellow lines show the region we choose to fit. The continuums are showed in blue.}
\label{fig:all_fit}
\end{figure*} 

\end{document}

% End of file `sample631.tex'.
